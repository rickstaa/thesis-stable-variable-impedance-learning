\chapter*{Abstract}
\addcontentsline{toc}{chapter}{Abstract}

This literature provides an extensive overview of the stability/passivity considerations and methods used in the current variable impedance literature for stable control of rigid robotics manipulators in contact with (semi)-rigid environments. Stability is essential for the safe operation of robotic systems in the real world, as unstable systems can exhibit unpredictable and dangerous behaviour. Impedance control, which models robot-environment interaction as a spring-damper system, is commonly used for safe interactions, but variable impedance parameters can cause instability. To solve this, researchers have used Lyapunov's theory and its extension passivity to ensure variable impedance control stays stable in free motion and is bounded during an interaction. Two passivity-based approaches are found within the literature for ensuring the stability of interaction tasks: pacifying control algorithms and stability constraints. Pacifying control algorithms, such as energy tanks and potential fields, rely on an energy-storing element to filter non-passive actions and protect the total system stability. They are easy to implement, offer clear physical insights and can be used with any controller. However, they require task-specific tuning and can only be used online, making them unsuitable for ensuring predetermined trajectories and impedance profiles. Stability constraints, conversely, can be used to ensure the passivity of a given controller or impedance profile offline, but deriving them is challenging. Additionally, their effectiveness depends on the choice of Lyapunov candidates or contractive metrics used during their derivation, giving rise to an accuracy-stability trade-off. Nonetheless, state-independent stability constraints have been derived for the general-impedance controller and have been utilized to ensure the stability of adaptive or optimal control architectures. Similar stability constraints have also been derived and applied to imitation-based impedance controllers. Reinforcement learning-based variable impedance control has limited research on passivity and stability. However, recent papers have used constraint-based methods to impose stability constraints on either the sampling policy or NN-based policies to ensure passivity and stability. As these approaches rely on deterministic policies, ensuring passivity and stability in stochastic policy-based VIC remains an open question. Although the stability methods discussed in this literature review are adequate for ensuring stability in interaction tasks, this review has several limitations. It does not address interactions with active or compliant environments, and it primarily focuses on passivity-based stability methods while other methods are available. Furthermore, safety constraints such as position, velocity, and acceleration limits must be considered to control robotic manipulators alongside humans or other robots safely. Future research can investigate how these safety constraints can be integrated into learned impedance control policies while maintaining passivity and stability.
