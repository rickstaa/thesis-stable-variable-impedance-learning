\chapter{Discussions and Conclusion}
\label{chapter:conclusion}

% TODO: Check safety references remove some.
% TODO: Vraag kan nog duidelijker terug komen en de because valt wat uit de lucht.
% IN the precious
The previous chapters gave an extensive overview of the stability/passivity considerations and methods used in the current variable impedance literature for stable control of rigid robotics manipulators in contact with (semi)-rigid environments. Because unstable systems can exhibit unpredictable and often dangerous behaviours guaranteeing stability is essential if we want robots to transition from their isolated factory workspace to the real world, where they interact with humans and other robots. Impedance control, in which the robot-environment interaction is modelled as a spring-damper system, is frequently used as traditional position and force controllers are unsuited for simultaneously performing free motion trajectory tracking and environment interaction. Although impedance controllers with constant parameters are passive and thus stable by design, this passivity is lost when the impedance parameters vary with time. To solve this, researchers have used Lyapunov's theory and its extension passivity to ensure VIC stays stable in free motion and is bounded during an interaction.

Within the variable impedance literature, two types of passivity-based approaches for ensuring the stability of interaction tasks are found: Solutions that try to enforce passivity online through pacifying control algorithms and solutions that use stability constraints to ensure the passivity of a given controller or impedance profile offline. Under the first approach (i.e. pacifying control algorithms), methods like energy tanks and potential fields are found. These methods rely on an energy-storing element (such as a tank or potential field) to track the energy dissipated in the closed-loop system. They act as a passivity filter by only executing non-passive actions when enough energy is left in this storage element. When no energy is left, non-passive actions can no longer be performed, thereby protecting the total system passivity. Several authors have used these storage-based methods with optimal, adaptive and LfD-based variable impedance controllers to ensure passivity and, thus, stability. They are easy to implement, offer clear physical insights, and can be used with any controller. However, they are task and state-dependent and must therefore be re-tuned for each task and can only be applied online. As a result, they are not suited for guaranteeing the execution of desired trajectories and impedance profiles beforehand. 

Therefore, other authors have taken the second approach and derived stability constraints that ensure the stability of their optimal and adaptive controllers. However, as no general method exists for deriving these constraints, this task is challenging. Furthermore, as these derivations prove stability and not passivity, they do not ensure these controllers stay stable when interacting with the environment. 

Kronander et al.

Extenstions

In learning based demonstration methods thoerica passivity guarnatees were dervied.



However, in the classical  impedance literature 

of optimal and adaptive controllers and the passivity of LfD-based variable impedance controllers. 



This was first done by Kronander et al. and later extended to also show passivity and closed loop stability.




Zowel binnen LFD als classical control Lyapunov constraints.
Maar geen general method dus challanging.
Kronder et al. derived these constraintes for the general impedance controller.
Later extended to also show passivity and therefore closed-loop stabiltiy.
However only one paper used these constraints online. Only specific problem.
During the optimization these constraints however lead to a accucart-stability trade-off.
Several authors have dervied less constricitve constrains improving performance. 
This however is an ongoing reaseach question.

Each method has its strengths and weaknesses.
Energy tanks can be best applied if you want to ensure passivity with controllers for which you don't know the impedance profiles.
Lyapunov consrains are used when you know these profiles beforehand.

is a challenging task as no general method exists for deriving these constraints, and the exact derivation depends on the controller and the environment.


Although these passivity constraints can ensure the passivity of a certain impedance profile or cotnroller beforehand . Furthermore both 
Further their  

These passivity constraint based methods, with the exception of one paper can only be used offline prior to the control 





Furthermore, their accuracy depends on the type of Lyapunov candidate or contraction metric used. 

Contrary to the storage-based methods, these constraint based methods are harder to implement 



Both 


For RL-based impedance controllers, studies that look at guaranteeing passivity are more sparse
. Nevertheless, some recent papers have looked at ways to ensure stability in these methods.


To the question of which method to use for your variable impedance controller . Since each methods above has its strenghts and weaknesses no easy answer is found to

Where 

As a result, no answer passivity method that is highly dependent on the type of impedance controller you are using and the task you are trying to achieve.

As shown above, both approaches have their strengths and weaknesses. As a result, no clear answer exists as to which passivity-based method is preferred since the answer depends on the type of impedance controller and the task you are trying to achieve. 

Constraint-based methods, on the other hand, can only be applied


Because both approaches have their

Within RL-based variable impedance control only some recent papers have looked at passivity. 


% Question: not the only two times?
However, this review has several limitations that need to be pointed out. First, it only considers interactions with rigid (passive) environments. It does not look at interactions with active or compliant environments like humans, controlled robots or soft objects. This review's passivity and ISS stability proofs do not hold for these environments. Therefore, ensuring stability in these environments is left for future research. Secondly, it mainly focuses on passivity-based stability methods. Even though passivity is well suited for guaranteeing stability in interaction tasks, it is not the only method. Other ISS and IOS methods are also used in the literature but fall outside this paper's scope. Lastly, while the controller's stability is trivial for a safe robot interaction, it is not the only quantity that needs to be considered. While a stable controller will not lead to unpredictable free motion and contact behaviour, it can still command high velocities, accelerations and forces, leading to unsafe situations. It should, therefore, also adhere to specific safety constraints like position, velocity, and acceleration limits to safely control robotic manipulators alongside humans or other robots \cite{lasotaSurveyMethodsSafe2017,raiolaDevelopmentSafetyEnergyAware2018,chowLyapunovbasedApproachSafe2018,zacharakiSafetyBoundsHuman2020,benziOptimizationApproachRobust2021,sadanandananandSafeLearningControl2021,sharkawyHumanRobotInteraction2022}. A possible future work could be to investigate how these additional safety constraints can be incorporated into the (learned) impedance control policy while maintaining passivity and stability.

% TODO: Tank based and potential field based need to be tuned to each task they apply to. Which method is to be used ultimately depends on factors such as importance of accurate stiffness following (choose our approach) or applicability to systems without dynamic decou-pling (choose state-dependent approach such as [8]) (kronander)

% TODO: Future researchers could focus on Lyapunov constraint always less conservative constraints can be found. Further constraints which teh stiffness or damping can be varied faster can be found (Sun and Zhang).

% TODO: potential fields only allow varying stiffness not applied yet to also include inertia variation.

% TODO: Admittance with position control passivity and stability can be easily proven (Kramberger, Tang et al.).

% TODO: DS not only modeling approach.

% TODO: Kronander et al. 2016 time dependent LFD based methods are state dependent.

--------------------------------

Within the first approach 

The type of passivity method used is highly dependent on the task that needs to be performed. Methods like energy tanks and potential fields can be applied to any to ensure control stays passive and thus stable. However, they are state-dependent and can significantly affect trajectory tracking performance. 

Passivity based methods like those of KRonander et al. 2014 on the other hand allow 

As stability and accuracy have a trade-off, people must look at the task to decide which is most important (see califano).

The type of passivity method depends on the task at hand. If you can verify the passivity of the task beforehand, a constraint can be used; otherwise, an energy-tank or potential field can be used.

Several methods have been proposed to ensure free-motion and interaction stability in variable impedance control:

Where online methods are used when the impedance profile or interaction force can not be predicted, constraint methods are used when the impedance profile is known in advance.

Authors have been improving these methods so that they are less restrictive and better tracking performance can be achieved. Future research should therefore focus on these methods and their improvements. 

With impedance-based algorithms, several methods have been used to ensure stable interaction and accurate trajectory-tracking performance. 

% Khansari potential has to be retrained when desired trajectory changes.
Using a fractal attractor is very promising for enforcing the passivity of variable impedance profiles online. Currently, this method has only been applied on variable impedance controllers in which only the stiffness has been varied since this is the most important for stability? (See article) Future research should focus on applying this concept to variable impedance controllers where all impedance parameters can vary.
Derive Lyapunov constraints in which the inertia does not have to be constant.
Multiple methods have been proposed to ensure robot manipulators' stability when interacting with rigid environments through VIC. Researchers used passivity filters like Farragut's energy tank to enforce the passivity of the closed-loop system during control \cite{ferragutiTankbasedApproachImpedance2013}. The benefit of this method is that it does not require any knowledge about the controlled system. The downside is that it is state-dependent and does not guarantee control performance. Researchers like \cite{kronanderStabilityConsiderationsVariable2016}have tried to solve these problems by designing passivity constraints on the variable impedance parameters that ensure a controller stays passive. Other authors have designed controllers, of which passivity is shown using a passivity analysis. Although this solves the state dependency problem and improves performance, this method is not always feasible since it requires in-depth knowledge of the dynamic system and uncertainties. Learning-based methods have been used to overcome the above shortcomings.

Passivity-based methods have been extensively used in impedance research to ensure stable control. Methods like energy

Each of these methods has its strengths and weaknesses. Where are used for are used when they are not known in advance.

The type of passivity solution depends on the task that needs to be performed. A proven passive controller like \cite{yangHumanlikeAdaptationForce2011} Yang et al. is preferable since this controller will always stay passive. Designing such a controller, however, is not trivial. For tasks where an offline profile is computed \cite{ferragutiTankbasedApproachImpedance2013} Ferragutis thank or the Lyapunov constraints of \cite{kronanderStabilityConsiderationsVariable2016} Kronander are very helpful. Learned controllers, on the other hand, do provide more customizability and are therefore preferred.

Search for even less conservative passivity constraints.

Although the use of FA to create FIC is up-and-coming concept for being able to use VIC. It however has not yet been applied for VIC in which also the desired inertia is varied and for which the (nonlinear) variable impedance profile is not set in advance and only . Future research should therefore focus on applying this concept to variable impedance controllers where all impedance parameters can vary and where the desired inertia is set in advance. assumption based on nonlinear profile.
Damping can be introduced as long as the desired velocity is zero.

The exact stability method used depends on the task you are trying to achieve. DMP for example, are very easy to implement and work for tatsks that have a clear end time and can be learned from one demonstration?

LfD is a trade-off between accuracy-stability and model/learning complexity.

- DMP easy to implement -> computationally efficient.
- Constraint methods -> more complex but more accurate.
- Learned constraint methods -> even more complex and time consuming but again more accurate.
- Diffeomophic methods -> give more acccurate results but require to learn diffeomorphisms
- contraction based methods -> Give trajectory tracking but can be over-constricting outside ot eh demonstrations.
- Energy tank -> straightforward but influences the accuracy online
