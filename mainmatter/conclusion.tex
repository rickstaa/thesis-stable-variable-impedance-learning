\chapter{Discussions and Conclusion}
\label{chapter:conclusion}

In the previous chapters, we saw that Lyapunov's stability and passivity are essential for ensuring stable control of rigid robotics manipulators in contact with (semi)-rigid environments. Because unstable systems can exhibit unpredictable and often dangerous behaviours guaranteeing stability is essential if we want robots to transition from their isolated factory workspace to the real world, where they interact with humans and other robots. Impedance control, in which the robot-environment interaction is modelled as a spring-damper system, is frequently used because traditional position or force controllers are unsuited for simultaneously performing free motion trajectory tracking and environment interaction. Although impedance controllers with constant parameters are passive and thus stable by design, this passivity is lost when the impedance parameters vary with time. To solve this, researchers have used Lyapunov's theory and its extension passivity to ensure VIC stays stable in free motion and is bounded during an interaction.

Within the variable impedance literature, two types of passivity-based approaches for ensuring the stability of interaction tasks are found: Solutions that try to enforce passivity online through pacifying control algorithms and solutions that use stability constraints to ensure the passivity of a given controller or impedance profile offline. Under the first approach (i.e. pacifying control algorithms), methods like energy tanks and potential fields are found. First proposed by \cite{ferragutiTankbasedApproachImpedance2013}, these methods rely on an energy-storing element (such as a tank or potential field) to track the energy dissipated in the closed-loop system. They act as a passivity filter by only allowing the execution of non-passive actions when enough energy is available in this storage element. When no energy is left, non-passive actions can no longer be performed, thereby protecting the total system passivity. Several authors have used these storage-based methods with optimal, adaptive and LfD-based variable impedance controllers to ensure passivity and stability. They are easy to implement, offer clear physical insights, and can be used with any controller. However, these methods are task-dependent and therefore require tuning of the storage parameters for each task they are applied. More importantly, they are also state-dependent and can only be applied online. As a result, they are not suited for guaranteeing the execution of desired trajectories and impedance profiles beforehand. Therefore, other authors have taken the second approach and derived stability constraints that ensure the stability of their optimal and adaptive controllers. However, as no general method exists for deriving these constraints, this task is challenging. Furthermore, as most papers only prove stability and not passivity, they do not ensure these controllers stay stable when interacting with the environment. As a result, Kronander et al. \cite{kronanderStabilityConsiderationsVariable2016} and follow-up papers, therefore, derived state-independent stability constraints on the general-impedance controller that can be used to check the stability and passivity of a given impedance profile offline or directly incorporated into an optimization or learning procedure. Multiple authors have used these constraints to ensure that impedance profiles from their adaptive or optimal control architectures are stable. Similar stability constraints were derived and used with LfD-based impedance controllers. These constraint-based methods provide a state-independent way to ensure the passivity and, thus, stability of offline impedance profiles while having less tunable parameters than the storage-based methods above. Nevertheless, except for a recent paper by Bednarcyk et al. \cite {bednarczykPassivityFilterVariable2020}, these constraint-based methods do not provide a way to modify non-passive impedance profiles online. More importantly, although they generally result in better tracking and impedance accuracy than storage-based methods, this performance highly depends on the Lyapunov candidates or contractive metrics used during the derivation giving rise to a so-called accuracy-stability trade-off. Although, since their initial derivation by \cite{kronanderStabilityConsiderationsVariable2016} less constrictive Lyapunov candidates have been found, this still is an ongoing area of research. Because of this, no clear answer can be given as to which passivity approach is best, as the answer is highly dependent on the exact controller architecture being used and the task that needs to be performed. Where storage-based methods can easily ensure passivity when the exact impedance profile and controller specifics are unknown, constraint-based methods are preferred when the desired impedance profiles are known beforehand.

This literature review tried to fill the gap in the current impedance literature regarding stability/passivity considerations and methods used for stable control of rigid robotics manipulators in contact with (semi)-rigid environments. However, it also has several limitations that need to be pointed out. First, it only considers interactions with rigid (passive) environments. It does not look at interactions with active or compliant environments like humans, controlled robots or soft objects. This review's passivity and ISS proofs do not hold for these environments. Therefore, ensuring stability in these conditions is left for future research. Secondly, it mainly focuses on passivity-based stability methods. Even though passivity is well suited for guaranteeing stability in interaction tasks, it is not the only method. Other ISS and IOS methods are also used in the literature but are beyond the scope of this literature review. Lastly, while the controller's stability is trivial for a safe robot interaction, it is not the only quantity that needs to be considered. While a stable controller will not lead to unpredictable free motion and contact behaviour, it can still command high velocities, accelerations and forces, leading to unsafe situations. It should, therefore, also adhere to specific safety constraints like position, velocity, and acceleration limits to safely control robotic manipulators alongside humans or other robots \cite{lasotaSurveyMethodsSafe2017,chowLyapunovbasedApproachSafe2018,sadanandananandSafeLearningControl2021,sharkawyHumanRobotInteraction2022}. A possible future area of research could be to investigate how these additional safety constraints can be incorporated into the (learned) impedance control policy while maintaining passivity and stability.

% TODO: Sometimes accuracy is more important and sometimes stability? Task dependence is this clear!

% TODO: Since passivity and stability constraints in RL-based impedance controllers are a relatively new field of research currently only Quadratic stability constraints are used to limit the exploration policy. Future research should therefore focus on applying Learned Lyapunov methods, contraction theory and diffeomorphism to RL-based methods could yield very promising results.

IN RL based research only simple Quadratic constraints are used to constraint the exploration. A future extension would be to apply the learning based/contracti

% TODO: Smooth chatter-free contact is not ivestigated (see khander et al. 2020)
