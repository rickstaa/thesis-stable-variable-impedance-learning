\chapter{Introduction}
\label{chapter:introduction}

As more and more robots move out of the factory into the real world, there is a growing need for robots that can safely work alongside humans or other robots \cite{sharkawySurveyApplicationsHumanRobot2021,zacharakiSafetyBoundsHuman2020,suomalainenSurveyRobotManipulation2021}. In order to do so, these robots must not only detect and avoid collisions \cite{haddadinRobotCollisionsSurvey2017, zacharakiSafetyBoundsHuman2020} but also safely interact with humans and the environment when a collision or a desired interaction occurs. Several researchers have proposed compliant or soft robots to prevent hard collisions and thus ensure safe interactions \cite{hughesSoftManipulatorsGrippers2016,douSoftRoboticManipulators2021}. Even though these robots are, without doubt, safer than their rigid counterparts, they are more mechanically complex and unsuitable for tasks requiring high position accuracy, repeatability and effort \cite{douSoftRoboticManipulators2021}. As a result, researchers mainly resort to using rigid manipulators (sometimes with soft grippers) for robot manipulation tasks.

Although traditional position, velocity, and acceleration control algorithms, often used with rigid manipulators, achieve high precision and repeatability, they are stiff and not sensitive to force interaction. As a result, they are not suited for interacting with the environment because they may lead to large contact forces, resulting in unstable behaviour, damage or even accidents. Effort control algorithms, on the other hand, can precisely regulate contact force but cannot accurately track a desired path without contact. To perform interaction tasks like manipulation, which contain both a free and contact phase, several researchers have used hybrid force/motion controllers \cite{raibertHybridPositionForce1981,ortenziHybridMotionForce2017}. These controllers divide the task space into a position-controlled (i.e., free) and force-controlled (constraint) subspace and switch between these spaces during the task. However, designing the switching behaviour of these hybrid controllers requires prior knowledge of the structure and geometry of the environment, which might not be available and is task-dependent also. Because of this, hybrid control algorithms cannot perform well in unstructured or dynamically changing environments.

Impedance control, in which the robot-environment interaction is modelled as a spring-damper system, is a better candidate. It can track a free-space trajectory while limiting the force applied to the environment when in contact. Traditionally, impedance controllers with constant parameters have been used \cite{hoganImpedanceControlApproach1985,hoganStableExecutionContact1987,calancaReviewAlgorithmsCompliant2016,songTutorialSurveyComparison2019,cheahLearningImpedanceControl1998,liHumanRobotCollaboration2013,heAdaptiveNeuralImpedance2015,liAdaptiveImpedanceControl2016,jamwalImpedanceControlIntrinsically2016,jungForceTrackingImpedance2004,ottPrioritizedMultitaskCompliance2015}. These controllers are passive and guarantee that the control never becomes unstable when controlling a passive robot in a passive environment. However, these controllers require accurate knowledge about the stiffness and location of the environment and can only attain a constant desired force. They can, therefore, not be used for dynamic force tracking or in uncertain or changing environments.

Variable impedance control (VIC) with time-varying impedance parameters provides a solution to this problem. By allowing the parameters to change over time, more complex tasks can be executed while improving robustness against uncertainties and changing environments. These variable impedance profiles can be implemented using adaptive or optimal controllers \cite{songTutorialSurveyComparison2019,erdenAssistingManualWelding2011,erdenRoboticAssistanceImpedance2016,leeForceTrackingImpedance2008,ikeuraOptimalVariableImpedance2002,medinaRisksensitiveInteractionControl2013,yangHumanlikeAdaptationForce2011} or learned from human demonstrations or through reinforcement learning (RL) \cite{buchliLearningVariableImpedance2011,rombokasTendondrivenVariableImpedance2013,michelBilateralTeleoperationAdaptive2021,chenClosedLoopVariableStiffness2021,weiImpedanceControlUncertain2019,houVariableImpedanceControl2020,abu-dakkaVariableImpedanceControl2020,calinonLearningbasedControlStrategy2010,kronanderOnlineLearningVarying2012}. Although promising results have been obtained using these controllers in surgery \cite{ferragutiEnergyTankBasedInteractive2015, ferragutiTankbasedApproachImpedance2013}, human-robot interaction \cite{wuAdaptiveImpedanceControl2020,san-miguelAutomatedOffLineGeneration2022,sharifiImpedanceLearningBasedAdaptive2021} and welding and grinding tasks \cite{erdenAssistingManualWelding2011,erdenRoboticAssistanceImpedance2016,zhangLearningImpedanceRegulation2021,wangSafeOnlineGain2021}, it introduced one big problem. By varying the impedance parameters, the passivity property of the system no longer holds as energy can now be injected into the system, possibly making it unstable \cite{ferragutiTankbasedApproachImpedance2013}. Because unstable systems exhibit unpredictable and uncontrollable behaviour, they can not be deployed safely alongside humans or other robots.

Several solutions have been proposed in the relevant literature for keeping the control passive with varying impedance parameters. Although in recent years, multiple reviews have been conducted on the current state of the literature regarding VIC algorithms that robotic manipulators use for interaction tasks, they only briefly touch on this passivity issue \cite{suomalainenSurveyRobotManipulation2021,songTutorialSurveyComparison2019,abu-dakkaVariableImpedanceControl2020,al-shukaActiveImpedanceControl2018}. Therefore, this literature review aims to fill this gap by providing an extensive overview of the stability/passivity considerations and methods used in the current variable impedance literature for stable control of rigid robotics manipulators in contact with \textbf{(semi)-rigid environments}. Even though most of the techniques discussed in this literature review can also be applied to VIC in human-robot interaction (HRI) tasks, it is not the primary focus of this review. Readers interested in that subject can read the review done by \cite{sharifiImpedanceVariationLearning2021}. This review also does not consider stability problems caused by time delays in the control loop; such is often encountered during teleoperation. An extensive review on this subject can be found in \cite{farajiparvarBriefSurveyTelerobotic2020}.

This literature review is structured as follows: chapter \ref{chapter:method} depicts the methodology used for finding the relevant literature used in this review. After that, in chapter \ref{chapter:background}, essential concepts like VIC, Lyapunov stability and passivity are reviewed. The currently used techniques for ensuring closed-loop stability of classical variable impedance controllers are presented in chapter \ref{chapter:classical_variable_impedance}, while chapter \ref{chapter:learning_based_variable_impedance} describes the stability techniques used in learning-based controllers. Finally, chapter \ref{chapter:conclusion} discusses the current research trends and future challenges.
